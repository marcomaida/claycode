\begin{abstract}
    % \noindent 
    % In this paper, we introduce Claycodes, a new type of scannable visual codes. Claycodes extend the functionalities of traditional visual codes (e.g., QR-Codes) by allowing an unprecedented
    % degree of customisation, both of the shape of the code and on its style, bridging the
    % world of designers with the needs of machine-readable visual codes.
    % The principle behind Claycode's visual encoding makes them the first visual code to be 
    % independent from their shape (i.e., not limited to a square, rectangle, or circle). 
    % Thanks to this property, they can be placed within any region (provided it is large enough
    % to contain the message). 
    % In this paper, we introduce Claycodes and illustrate their novel bit encoding. 
    % Next, we propose an algorithm to efficiently generate a Claycode given a fitting shape and a message,
    % and show how the message can be decoded. 
    % Finally, we compare Claycodes with other scannable visual codes in terms of space efficiency.
\end{abstract}