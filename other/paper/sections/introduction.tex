\section{Introduction}

Similarly to QR codes, some aesthetic codes such as ARTcodes \cite{artcodes} use a matrix to encode the data. Other authors have employed steganography \dots

Topological markers encode information in the structure of the image rather than its pixel-level content. Examples of this approach include ReacTIVision \cite{reactivision}, ARTag \cite{ARTag}, D-touch \cite{dtouch0}\cite{dtouch1}\cite{dtouch2}, and Seedmarkers \cite{seedmarkers}. Previous work has explored how this technology can be used in practice to create aesthetic interactive scannable objects such as tableware \cite{tableware} or speakers \cite{seedmarkers}, and digital media such as wallpapers \cite{interactiveWallpapers}.

One of the major challenges for aesthetic markers is ensuring that the marker can be recognized as such by users \cite{recognizingPresence}. In this paper, we address this issue both in the case where the raw data is encoded (section X), and when the data is defined by the art (section Y).