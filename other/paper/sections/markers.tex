\section{Claycode Markers}

In this section we explore an alternative application of Claycodes: rather than encoding data in the code itself, we generate a topology starting from an arbitrary image and use it as the code. Under this design, the actual data is stored elsewhere (e.g. in a cloud application) while the code itself serves as a key to the data.

It has been shown in practice that the requirement for an internet connection does not hinder the adoption of a scanner application. In the case of Navilens \cite{navilens} \dots

\begin{figure}[h]
    \caption{Revised encoding-decoding pipeline.}
    \centering
    \includegraphics[width=0.45\textwidth]{markers}
\end{figure}

\subsection[artDefined]{Art-Defined Markers}

This approach allows the markers to be stylizable with minimal constraint on the artist.

\subsection[fragments]{Fragmented Markers}

Using a dedicated editor, the artist can choose which areas are dedicated to the main subject, and which areas are dedicated to the floating fragments. Previous work has explored the idea of using an editor to help artists draw specific data while not violating the constraints of the marker \cite{intentionalEncoding}. Similarly to art-defined markers, this approach ensures that the whole image is stylizable, but it increases the burden on the artist and does not achieve error correction.

To achieve error correction based on redundancy, we can use a similar approach to instead generate multiple topologies embedded in an arbitrary image. This type of error correction is resistant to certain classes of tampering that QR codes are vulnerable to.
